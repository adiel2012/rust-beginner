\documentclass[11pt,a4paper]{book}
\usepackage[utf8]{inputenc}
\usepackage[margin=1in]{geometry}
\usepackage{listings}
\usepackage{xcolor}
\usepackage{hyperref}
\usepackage{fancyhdr}

% Page style
\pagestyle{fancy}
\fancyhf{}
\fancyhead[LE,RO]{\thepage}
\fancyhead[RE]{\leftmark}
\fancyhead[LO]{\rightmark}

% Rust code highlighting
\lstdefinelanguage{Rust}{
  keywords={fn, let, mut, if, else, match, for, while, loop, impl, struct, enum, trait, pub, use, mod, return, break, continue, const, static, type, where, self, Self, async, await, unsafe, dyn, Box, Vec, String, Option, Result, Some, None, Ok, Err},
  keywordstyle=\color{blue}\bfseries,
  ndkeywords={i32, i64, u32, u64, f32, f64, bool, char, str, usize},
  ndkeywordstyle=\color{purple}\bfseries,
  sensitive=true,
  comment=[l]{//},
  morecomment=[s]{/*}{*/},
  commentstyle=\color{gray}\ttfamily,
  stringstyle=\color{red}\ttfamily,
  morestring=[b]',
  morestring=[b]"
}

\lstset{
  language=Rust,
  basicstyle=\ttfamily\footnotesize,
  breaklines=true,
  frame=single,
  numbers=left,
  numberstyle=\tiny\color{gray},
  showstringspaces=false,
  captionpos=b
}

\title{Advanced Rust Programming\\
\large Expert-Level Techniques and Internals}
\author{Rust Learning Series - Advanced Track}
\date{\today}

\begin{document}

\maketitle

\chapter*{Prerequisites}
\addcontentsline{toc}{chapter}{Prerequisites}

This book assumes mastery of all intermediate Rust concepts. You should have extensive practical experience with:

\begin{itemize}
\item \textbf{All Intermediate Topics}: Advanced traits, complex lifetimes, and macro programming
\item \textbf{Unsafe Rust and FFI}: Comfortable writing unsafe code and interfacing with C
\item \textbf{Advanced Async Programming}: Understanding futures, pinning, and async runtimes
\item \textbf{Performance Optimization}: Experience with profiling and optimization techniques
\item \textbf{Lock-Free Concurrency}: Understanding atomic operations and memory orderings
\end{itemize}

This book covers expert-level topics including compiler internals, advanced procedural macros, custom allocators, embedded systems programming, WebAssembly, async runtime implementation, advanced lock-free algorithms, language tooling development, performance profiling, and contributing to Rust itself.

\tableofcontents

% ============================================
% CHAPTER 1: COMPILER INTERNALS AND MIR
% ============================================
\chapter{Compiler Internals and MIR}

Understanding Rust's compilation pipeline provides insights into how the language works and enables advanced compiler development.

\section{The Rust Compilation Pipeline}

The Rust compiler (rustc) transforms source code into machine code through several intermediate representations:

\begin{enumerate}
\item \textbf{Lexing and Parsing}: Source code is tokenized and parsed into an Abstract Syntax Tree (AST)
\item \textbf{Macro Expansion}: Macros are expanded to generate more AST nodes
\item \textbf{HIR (High-level IR)}: AST is lowered to a more compiler-friendly representation
\item \textbf{Type Checking and Inference}: Type information is inferred and checked
\item \textbf{MIR (Mid-level IR)}: HIR is lowered to MIR for borrow checking and optimization
\item \textbf{Borrow Checking}: MIR is analyzed to enforce borrowing rules
\item \textbf{Optimization}: Various optimizations are applied to MIR
\item \textbf{LLVM IR}: MIR is translated to LLVM's intermediate representation
\item \textbf{Machine Code}: LLVM generates architecture-specific machine code
\end{enumerate}

\section{Working with MIR}

Mid-level Intermediate Representation (MIR) is crucial for borrow checking and optimization. You can inspect MIR for your functions:

\begin{lstlisting}
// View MIR for a function
#[rustc_dump_mir(before = "all", after = "all")]
fn example(x: i32) -> i32 {
    x * 2 + 1
}
\end{lstlisting}

\subsection{MIR Structure}

MIR represents functions as control flow graphs with basic blocks:

\begin{lstlisting}
// Simplified MIR representation
fn example(_1: i32) -> i32 {
    let mut _0: i32;
    let mut _2: i32;
    let mut _3: i32;

    bb0: {
        _2 = _1;
        _3 = const 2_i32;
        _2 = Mul(move _2, move _3);
        _3 = const 1_i32;
        _0 = Add(move _2, move _3);
        return;
    }
}
\end{lstlisting}

MIR is used for:
\begin{itemize}
\item \textbf{Borrow Checking}: Ensuring ownership and borrowing rules
\item \textbf{Optimization}: Performing transformations that preserve semantics
\item \textbf{Const Evaluation}: Computing compile-time constants
\end{itemize}

\section{Compiler Plugins and Custom Lints}

You can extend rustc with custom lints:

\begin{lstlisting}
#![feature(rustc_private)]
extern crate rustc_lint;
extern crate rustc_middle;

use rustc_lint::{LateContext, LateLintPass, LintContext};

declare_lint! {
    pub CUSTOM_LINT,
    Warn,
    "description of custom lint"
}

impl<'tcx> LateLintPass<'tcx> for CustomLint {
    fn check_expr(&mut self, cx: &LateContext<'tcx>, expr: &Expr) {
        // Custom lint logic
        // Analyze expressions and emit warnings
    }
}
\end{lstlisting}

Custom lints enable enforcement of project-specific coding standards and detection of anti-patterns.

% ============================================
% CHAPTER 2: ADVANCED PROCEDURAL MACROS
% ============================================
\chapter{Advanced Procedural Macros}

Procedural macros enable powerful compile-time metaprogramming. Advanced techniques allow you to create complex code generation.

\section{Complex Derive Macros}

Derive macros automatically implement traits. Building sophisticated derive macros requires parsing attributes and generating complex code:

\begin{lstlisting}
#[proc_macro_derive(Builder, attributes(builder))]
pub fn derive_builder(input: TokenStream) -> TokenStream {
    let input = parse_macro_input!(input as DeriveInput);

    let name = &input.ident;
    let builder_name = format_ident!("{}Builder", name);

    let fields = match &input.data {
        Data::Struct(data) => &data.fields,
        _ => panic!("Builder only works on structs"),
    };

    // Generate builder methods for each field
    let methods = fields.iter().map(|f| {
        let field_name = &f.ident;
        let field_type = &f.ty;
        quote! {
            pub fn #field_name(mut self, value: #field_type) -> Self {
                self.#field_name = Some(value);
                self
            }
        }
    });

    // Generate the builder struct and implementation
    // ... more complex logic
}
\end{lstlisting}

\section{Parsing Custom Syntax}

The \texttt{syn} crate enables parsing arbitrary syntax:

\begin{lstlisting}
use syn::{parse::Parse, Token};

struct MyMacroInput {
    name: Ident,
    _arrow: Token![=>],
    value: Expr,
}

impl Parse for MyMacroInput {
    fn parse(input: ParseStream) -> Result<Self> {
        Ok(MyMacroInput {
            name: input.parse()?,
            _arrow: input.parse()?,
            value: input.parse()?,
        })
    }
}

#[proc_macro]
pub fn my_macro(input: TokenStream) -> TokenStream {
    let parsed = parse_macro_input!(input as MyMacroInput);
    // Use parsed.name and parsed.value to generate code
}
\end{lstlisting}

\section{Hygiene and Span Management}

Macro hygiene prevents identifier conflicts. Span management enables precise error reporting:

\begin{lstlisting}
use proc_macro::Span;
use quote::quote_spanned;

fn generate_with_span(span: Span) -> TokenStream {
    quote_spanned! {span=>
        compile_error!("Error at original location");
    }.into()
}

// Preserve spans for better error messages
let tokens = quote_spanned! {field.span()=>
    self.#field_name = value;
};
\end{lstlisting}

Macro-generated identifiers automatically avoid conflicts with user code, but you can control scoping when necessary.

% ============================================
% CHAPTER 3: CUSTOM ALLOCATORS
% ============================================
\chapter{Custom Allocators}

Custom allocators provide control over memory management for specialized use cases.

\section{The GlobalAlloc Trait}

Implement custom allocators using \texttt{GlobalAlloc}:

\begin{lstlisting}
use std::alloc::{GlobalAlloc, Layout, System};

struct MyAllocator;

unsafe impl GlobalAlloc for MyAllocator {
    unsafe fn alloc(&self, layout: Layout) -> *mut u8 {
        println!("Allocating {} bytes", layout.size());
        System.alloc(layout)
    }

    unsafe fn dealloc(&self, ptr: *mut u8, layout: Layout) {
        println!("Deallocating {} bytes", layout.size());
        System.dealloc(ptr, layout)
    }
}

#[global_allocator]
static ALLOCATOR: MyAllocator = MyAllocator;
\end{lstlisting}

\section{Arena Allocators}

Arena allocators provide fast bulk allocation and deallocation:

\begin{lstlisting}
struct Arena {
    buf: Vec<u8>,
    offset: usize,
}

impl Arena {
    fn alloc<T>(&mut self, value: T) -> &mut T {
        let layout = Layout::new::<T>();
        let offset = self.offset;
        self.offset += layout.size();

        unsafe {
            let ptr = self.buf.as_mut_ptr().add(offset) as *mut T;
            ptr.write(value);
            &mut *ptr
        }
    }

    fn reset(&mut self) {
        self.offset = 0;
        // No individual drops - bulk deallocation
    }
}
\end{lstlisting}

Arena allocators excel when you allocate many objects with similar lifetimes and deallocate them all at once.

\section{Per-Collection Allocators}

The \texttt{allocator\_api} feature enables per-collection custom allocators:

\begin{lstlisting}
#![feature(allocator_api)]

use std::alloc::Allocator;

struct BumpAllocator { /* ... */ }

unsafe impl Allocator for BumpAllocator {
    fn allocate(&self, layout: Layout)
        -> Result<NonNull<[u8]>, AllocError>
    {
        // Bump allocation logic
    }

    unsafe fn deallocate(&self, ptr: NonNull<u8>, layout: Layout) {
        // No-op for bump allocator
    }
}

// Use with collections
let vec: Vec<i32, BumpAllocator> = Vec::new_in(bump_allocator);
\end{lstlisting}

% ============================================
% CHAPTER 4: NO-STD AND EMBEDDED SYSTEMS
% ============================================
\chapter{No-Std and Embedded Systems}

Rust's no-std mode enables bare-metal programming for embedded systems and operating system kernels.

\section{No-Std Fundamentals}

In no-std environments, you don't have access to the standard library:

\begin{lstlisting}
#![no_std]
#![no_main]

use core::panic::PanicInfo;

#[panic_handler]
fn panic(_info: &PanicInfo) -> ! {
    loop {}
}

#[no_mangle]
pub extern "C" fn _start() -> ! {
    // Entry point for bare metal
    loop {}
}

// Use core instead of std
use core::ptr;
use core::mem;
\end{lstlisting}

No-std environments have no heap allocation, no operating system services, and minimal runtime support.

\section{Embedded HAL}

The Hardware Abstraction Layer enables portable embedded code:

\begin{lstlisting}
use embedded_hal::digital::v2::OutputPin;

struct Led<P: OutputPin> {
    pin: P,
}

impl<P: OutputPin> Led<P> {
    fn on(&mut self) -> Result<(), P::Error> {
        self.pin.set_high()
    }

    fn off(&mut self) -> Result<(), P::Error> {
        self.pin.set_low()
    }
}

// Works with any GPIO implementation
let mut led = Led { pin: gpio_pin };
led.on().unwrap();
\end{lstlisting}

\section{Volatile Memory Access}

Memory-mapped I/O requires volatile operations:

\begin{lstlisting}
use core::ptr::{read_volatile, write_volatile};

// Memory-mapped IO
const GPIO_BASE: usize = 0x4000_0000;

#[repr(C)]
struct GpioRegisters {
    data: u32,
    direction: u32,
    interrupt: u32,
}

fn set_gpio(bit: u8) {
    unsafe {
        let gpio = GPIO_BASE as *mut GpioRegisters;
        let mut data = read_volatile(&(*gpio).data);
        data |= 1 << bit;
        write_volatile(&mut (*gpio).data, data);
    }
}
\end{lstlisting}

Volatile operations prevent compiler optimizations that would be incorrect for hardware registers.

\section{Interrupt Handling}

Embedded systems use interrupts for asynchronous events:

\begin{lstlisting}
use cortex_m_rt::interrupt;

#[interrupt]
fn TIM2() {
    // Timer 2 interrupt handler
    static mut COUNT: u32 = 0;

    unsafe {
        *COUNT += 1;
        // Clear interrupt flag
        (*TIM2::ptr()).sr.modify(|_, w| w.uif().clear_bit());
    }
}

// Configure interrupt
unsafe {
    cortex_m::peripheral::NVIC::unmask(Interrupt::TIM2);
}
\end{lstlisting}

% ============================================
% CHAPTER 5: WEBASSEMBLY DEEP DIVE
% ============================================
\chapter{WebAssembly Deep Dive}

WebAssembly enables running Rust code in web browsers with near-native performance.

\section{Wasm Bindgen}

The \texttt{wasm-bindgen} tool generates JavaScript bindings:

\begin{lstlisting}
use wasm_bindgen::prelude::*;

#[wasm_bindgen]
pub fn fibonacci(n: u32) -> u32 {
    match n {
        0 => 0,
        1 => 1,
        _ => fibonacci(n - 1) + fibonacci(n - 2),
    }
}

#[wasm_bindgen]
extern "C" {
    #[wasm_bindgen(js_namespace = console)]
    fn log(s: &str);
}

#[wasm_bindgen]
pub fn greet(name: &str) {
    log(&format!("Hello, {}!", name));
}
\end{lstlisting}

\section{JavaScript Interoperability}

The \texttt{web-sys} crate provides Web API bindings:

\begin{lstlisting}
use wasm_bindgen::JsCast;
use web_sys::{Document, Element, HtmlElement};

#[wasm_bindgen(start)]
pub fn main() -> Result<(), JsValue> {
    let window = web_sys::window().unwrap();
    let document = window.document().unwrap();

    let body = document.body().unwrap();
    let div = document.create_element("div")?;
    div.set_inner_html("Hello from Rust!");

    body.append_child(&div)?;
    Ok(())
}
\end{lstlisting}

\section{Optimizing WebAssembly Size}

Size optimization is crucial for web delivery:

\begin{lstlisting}
# Cargo.toml
[profile.release]
opt-level = "z"          # Optimize for size
lto = true               # Link-time optimization
codegen-units = 1        # Better optimization
panic = "abort"          # Smaller binary
strip = true             # Strip symbols
\end{lstlisting}

Additional techniques:
\begin{itemize}
\item Avoid formatting macros in release builds
\item Use \texttt{wee\_alloc} for smaller allocator
\item Tree-shake with \texttt{wasm-gc}
\item Post-process with \texttt{wasm-opt}
\end{itemize}

% ============================================
% CHAPTER 6: ASYNC RUNTIME INTERNALS
% ============================================
\chapter{Async Runtime Internals}

Building custom async runtimes provides deep understanding of how async works in Rust.

\section{Building a Simple Executor}

An executor polls futures to completion:

\begin{lstlisting}
use std::future::Future;
use std::task::{Context, Poll, RawWaker, RawWakerVTable, Waker};

struct SimpleExecutor {
    tasks: Vec<Pin<Box<dyn Future<Output = ()>>>>,
}

impl SimpleExecutor {
    fn run(&mut self) {
        while !self.tasks.is_empty() {
            self.tasks.retain_mut(|task| {
                let waker = create_waker();
                let mut cx = Context::from_waker(&waker);

                match task.as_mut().poll(&mut cx) {
                    Poll::Ready(()) => false,  // Remove
                    Poll::Pending => true,     // Keep
                }
            });
        }
    }
}
\end{lstlisting}

\section{Waker Implementation}

Wakers notify the executor when futures are ready:

\begin{lstlisting}
fn create_waker() -> Waker {
    unsafe fn clone(ptr: *const ()) -> RawWaker {
        RawWaker::new(ptr, &VTABLE)
    }

    unsafe fn wake(_: *const ()) {
        // Wake the task
    }

    unsafe fn wake_by_ref(_: *const ()) {
        // Wake without consuming
    }

    unsafe fn drop(_: *const ()) {}

    static VTABLE: RawWakerVTable = RawWakerVTable::new(
        clone, wake, wake_by_ref, drop
    );

    let raw = RawWaker::new(std::ptr::null(), &VTABLE);
    unsafe { Waker::from_raw(raw) }
}
\end{lstlisting}

\section{Reactor Pattern}

Reactors handle I/O events:

\begin{lstlisting}
use mio::{Events, Interest, Poll, Token};

struct Reactor {
    poll: Poll,
    events: Events,
    handlers: HashMap<Token, Box<dyn FnMut()>>,
}

impl Reactor {
    fn register<S: Source>(&mut self, source: &mut S,
                           handler: impl FnMut() + 'static)
    {
        let token = Token(self.handlers.len());
        self.poll.registry()
            .register(source, token, Interest::READABLE)
            .unwrap();
        self.handlers.insert(token, Box::new(handler));
    }

    fn run(&mut self) {
        loop {
            self.poll.poll(&mut self.events, None).unwrap();
            for event in &self.events {
                if let Some(handler) = self.handlers.get_mut(&event.token()) {
                    handler();
                }
            }
        }
    }
}
\end{lstlisting}

% ============================================
% CHAPTER 7: ADVANCED LOCK-FREE ALGORITHMS
% ============================================
\chapter{Advanced Lock-Free Algorithms}

Lock-free data structures enable high-performance concurrent programming without locks.

\section{The ABA Problem}

Lock-free algorithms using compare-and-swap can suffer from the ABA problem:

\begin{lstlisting}
use std::sync::atomic::{AtomicPtr, AtomicUsize, Ordering};

struct Node<T> {
    data: T,
    next: *mut Node<T>,
}

struct Stack<T> {
    head: AtomicPtr<Node<T>>,
}

impl<T> Stack<T> {
    fn push(&self, data: T) {
        let node = Box::into_raw(Box::new(Node {
            data,
            next: std::ptr::null_mut(),
        }));

        loop {
            let head = self.head.load(Ordering::Acquire);
            unsafe { (*node).next = head; }

            if self.head.compare_exchange(
                head, node,
                Ordering::Release, Ordering::Acquire
            ).is_ok() {
                break;
            }
        }
    }
}
\end{lstlisting}

\section{Tagged Pointers}

Tagged pointers solve the ABA problem by including a version counter:

\begin{lstlisting}
// Pack counter with pointer
struct Tagged<T> {
    ptr: usize,  // Bottom bits: counter, top bits: pointer
}

impl<T> Tagged<T> {
    fn new(ptr: *mut T, tag: usize) -> Self {
        let addr = ptr as usize;
        Tagged { ptr: addr | (tag & 0xFFFF) }
    }

    fn get_ptr(&self) -> *mut T {
        (self.ptr & !0xFFFF) as *mut T
    }

    fn get_tag(&self) -> usize {
        self.ptr & 0xFFFF
    }
}

// Use AtomicUsize for tagged pointer
struct AbaFreeStack<T> {
    head: AtomicUsize,
}
\end{lstlisting}

\section{Epoch-Based Reclamation}

The \texttt{crossbeam-epoch} crate provides safe memory reclamation:

\begin{lstlisting}
use crossbeam_epoch::{self as epoch, Atomic, Owned};

struct Node<T> {
    data: T,
    next: Atomic<Node<T>>,
}

struct Stack<T> {
    head: Atomic<Node<T>>,
}

impl<T> Stack<T> {
    fn push(&self, data: T) {
        let node = Owned::new(Node {
            data,
            next: Atomic::null(),
        });

        let guard = epoch::pin();
        loop {
            let head = self.head.load(Ordering::Acquire, &guard);
            node.next.store(head, Ordering::Relaxed);

            if self.head.compare_exchange(
                head, node,
                Ordering::Release, Ordering::Acquire, &guard
            ).is_ok() {
                break;
            }
        }
    }
}
\end{lstlisting}

% ============================================
% CHAPTER 8: BUILDING LANGUAGE TOOLING
% ============================================
\chapter{Building Language Tooling}

Creating tools for Rust development enhances the ecosystem and provides deep language understanding.

\section{Using rust-analyzer APIs}

The \texttt{rust-analyzer} APIs enable code analysis:

\begin{lstlisting}
use ra_ap_syntax::{ast, AstNode, SyntaxKind};
use ra_ap_ide::{Analysis, AnalysisHost, FileId};

fn analyze_code(source: &str) -> Vec<String> {
    let parse = ast::SourceFile::parse(source);
    let root = parse.tree();

    let mut functions = Vec::new();

    for node in root.syntax().descendants() {
        if let Some(func) = ast::Fn::cast(node) {
            if let Some(name) = func.name() {
                functions.push(name.to_string());
            }
        }
    }

    functions
}
\end{lstlisting}

\section{Custom Cargo Subcommands}

Cargo subcommands extend Cargo's functionality:

\begin{lstlisting}
// cargo-mycmd/src/main.rs
use clap::Parser;

#[derive(Parser)]
#[command(name = "cargo")]
#[command(bin_name = "cargo")]
enum Cargo {
    Mycmd(Args),
}

#[derive(Parser)]
struct Args {
    #[arg(long)]
    verbose: bool,
}

fn main() {
    let Cargo::Mycmd(args) = Cargo::parse();

    // Access cargo metadata
    let metadata = cargo_metadata::MetadataCommand::new()
        .exec()
        .unwrap();

    // Implement custom logic
}
\end{lstlisting}

Install as \texttt{cargo-mycmd} and invoke with \texttt{cargo mycmd}.

\section{LSP Server Implementation}

Language Server Protocol enables IDE integration:

\begin{lstlisting}
use tower_lsp::{LspService, Server};
use tower_lsp::lsp_types::*;

struct Backend;

#[tower_lsp::async_trait]
impl LanguageServer for Backend {
    async fn initialize(&self, _: InitializeParams)
        -> Result<InitializeResult>
    {
        Ok(InitializeResult {
            capabilities: ServerCapabilities {
                text_document_sync: Some(
                    TextDocumentSyncCapability::Kind(
                        TextDocumentSyncKind::FULL
                    )
                ),
                completion_provider: Some(CompletionOptions::default()),
                ..Default::default()
            },
            ..Default::default()
        })
    }

    async fn completion(&self, _: CompletionParams)
        -> Result<Option<CompletionResponse>>
    {
        // Provide completions
    }
}
\end{lstlisting}

% ============================================
% CHAPTER 9: PERFORMANCE PROFILING
% ============================================
\chapter{Performance Profiling and Optimization}

Profiling identifies bottlenecks and guides optimization efforts.

\section{Benchmarking with Criterion}

Criterion provides statistical benchmarking:

\begin{lstlisting}
use criterion::{black_box, criterion_group, criterion_main, Criterion};

fn fibonacci_benchmark(c: &mut Criterion) {
    c.bench_function("fib 20", |b| {
        b.iter(|| fibonacci(black_box(20)))
    });

    c.bench_function("fib_iterative 20", |b| {
        b.iter(|| fib_iterative(black_box(20)))
    });
}

criterion_group!(benches, fibonacci_benchmark);
criterion_main!(benches);
\end{lstlisting}

Criterion features:
\begin{itemize}
\item Statistical analysis
\item HTML reports with plots
\item Comparison across runs
\end{itemize}

\section{CPU Profiling}

Various tools provide CPU profiling:

\begin{verbatim}
# Using perf on Linux
$ cargo build --release
$ perf record --call-graph=dwarf ./target/release/myapp
$ perf report

# Using flamegraph
$ cargo install flamegraph
$ cargo flamegraph

# Using samply (modern alternative)
$ cargo install samply
$ samply record ./target/release/myapp
\end{verbatim}

\subsection{Profile-Guided Optimization}

PGO optimizes based on actual runtime profiles:

\begin{verbatim}
# 1. Build with instrumentation
RUSTFLAGS="-Cprofile-generate=/tmp/pgo" cargo build --release

# 2. Run typical workloads
./target/release/myapp < typical_input.txt

# 3. Rebuild with profile data
RUSTFLAGS="-Cprofile-use=/tmp/pgo" cargo build --release
\end{verbatim}

\section{Memory Profiling}

Memory profiling identifies allocation patterns:

\begin{lstlisting}
// Using dhat for heap profiling
#[global_allocator]
static ALLOC: dhat::Alloc = dhat::Alloc;

fn main() {
    let _profiler = dhat::Profiler::new_heap();

    // Your code here
    let v: Vec<u64> = (0..1_000_000).collect();

    // Profiler drops, generating report
}
\end{lstlisting}

Valgrind provides another approach:

\begin{verbatim}
$ valgrind --tool=massif ./target/release/myapp
$ ms_print massif.out.12345
\end{verbatim}

% ============================================
% CHAPTER 10: CONTRIBUTING TO RUST
% ============================================
\chapter{Contributing to Rust Itself}

Contributing to Rust deepens your understanding and benefits the entire ecosystem.

\section{Rust Compiler Development}

Building rustc from source:

\begin{verbatim}
# Clone and build rustc
$ git clone https://github.com/rust-lang/rust.git
$ cd rust
$ ./x.py build

# Run tests
$ ./x.py test

# Build specific component
$ ./x.py build library/std

# Build documentation
$ ./x.py doc
\end{verbatim}

Key areas for contribution:
\begin{itemize}
\item \textbf{Compiler}: Type checking, borrow checking, MIR
\item \textbf{Standard Library}: Core, alloc, std
\item \textbf{Cargo}: Build system and package manager
\item \textbf{Rustdoc}: Documentation generator
\item \textbf{Clippy}: Linter
\end{itemize}

\section{The RFC Process}

Rust uses RFCs (Requests for Comments) for significant changes:

\begin{enumerate}
\item \textbf{Idea}: Discuss on internals.rust-lang.org
\item \textbf{RFC}: Submit RFC with detailed design
\item \textbf{Discussion}: Community reviews and suggests changes
\item \textbf{FCP}: Final Comment Period (10 days)
\item \textbf{Merge}: RFC accepted, implementation begins
\item \textbf{Implementation}: Write code, tests, documentation
\item \textbf{Stabilization}: Feature gate removal after testing
\end{enumerate}

\section{Writing Compiler Tests}

Rustc uses comprehensive test suites:

\begin{lstlisting}
// tests/ui/my-feature.rs
fn main() {
    let x: i32 = "hello"; //~ ERROR mismatched types
}
\end{lstlisting}

Running tests:

\begin{verbatim}
# Run specific test
$ ./x.py test tests/ui/my-feature.rs

# Update expected output
$ ./x.py test tests/ui --bless
\end{verbatim}

Test types:
\begin{itemize}
\item \texttt{ui}: Compiler error/warning tests
\item \texttt{compile-fail}: Tests that should fail compilation
\item \texttt{run-pass}: Tests that should compile and run
\item \texttt{incremental}: Incremental compilation tests
\end{itemize}

% ============================================
% CHAPTER 11: SUMMARY AND MASTERY PATH
% ============================================
\chapter{Summary and Mastery Path}

\section{Key Takeaways}

This book has covered expert-level Rust topics:

\begin{enumerate}
\item \textbf{Compiler Internals}: Understanding MIR and the compilation pipeline
\item \textbf{Procedural Macros}: Mastering complex code generation
\item \textbf{Custom Allocators}: Implementing specialized memory management
\item \textbf{No-Std Programming}: Building for embedded and bare metal
\item \textbf{WebAssembly}: Deploying Rust to the web
\item \textbf{Async Runtimes}: Building custom executors and reactors
\item \textbf{Lock-Free Algorithms}: Implementing advanced concurrent data structures
\item \textbf{Language Tooling}: Creating tools for Rust development
\item \textbf{Performance Profiling}: Optimizing for maximum performance
\item \textbf{Contributing to Rust}: Giving back to the ecosystem
\end{enumerate}

\section{Mastery Projects}

Challenge yourself with expert-level projects:

\begin{itemize}
\item Build a custom async runtime
\item Implement a garbage collector
\item Create a programming language in Rust
\item Write an operating system kernel
\item Build a database engine from scratch
\item Implement a JIT compiler
\item Create embedded firmware for real hardware
\item Contribute features to rustc or cargo
\end{itemize}

\section{Advanced Resources}

\subsection{Books and Documentation}

\begin{itemize}
\item \textbf{Rustc Dev Guide}: \texttt{rustc-dev-guide.rust-lang.org}
\item \textbf{Embedded Rust Book}: \texttt{docs.rust-embedded.org}
\item \textbf{Rust and WebAssembly Book}: \texttt{rustwasm.github.io}
\item \textbf{Rust Forge}: \texttt{forge.rust-lang.org}
\end{itemize}

\subsection{Research and Papers}

\begin{itemize}
\item Lock-free algorithms and data structures
\item Memory models and concurrency
\item Type system theory
\item Compiler optimization techniques
\end{itemize}

\subsection{Communities}

\begin{itemize}
\item \textbf{Zulip}: \texttt{rust-lang.zulipchat.com} - Real-time chat
\item \textbf{Internals Forum}: \texttt{internals.rust-lang.org} - Design discussions
\item \textbf{GitHub}: \texttt{github.com/rust-lang} - Source code
\item \textbf{Working Groups}: Async, embedded, WebAssembly, compiler
\item \textbf{This Week in Rust}: Weekly newsletter
\end{itemize}

\section{The Complete Journey}

You've completed the full Rust learning path from beginner to expert:

\begin{itemize}
\item \textbf{Beginner}: Fundamentals of ownership, types, and basic patterns
\item \textbf{Intermediate}: Advanced traits, lifetimes, macros, and async
\item \textbf{Advanced}: Compiler internals, systems programming, and ecosystem contribution
\end{itemize}

\section{Conclusion}

Congratulations on reaching expert level! You're now equipped to:
\begin{itemize}
\item Build production systems in Rust
\item Contribute to the Rust compiler and ecosystem
\item Tackle the most challenging programming problems
\item Mentor others in their Rust journey
\end{itemize}

The Rust community is welcoming and collaborative. Whether you're optimizing performance-critical code, building reliable systems, exploring language design, or teaching others, your expertise is valuable.

\textbf{Keep exploring, keep building, and keep contributing!}

\end{document}
